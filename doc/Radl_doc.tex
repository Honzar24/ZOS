\documentclass[12pt, a4paper]{report}
\usepackage[utf8]{inputenc}
\usepackage[IL2]{fontenc}
\usepackage[czech]{babel}
\usepackage[hidelinks]{hyperref}
\RequirePackage[a4paper,left=37mm,right=37mm,top=33mm,bottom=40mm]{geometry}

\usepackage{import}
\usepackage{xifthen}
\usepackage{pdfpages}
\usepackage{transparent}
\usepackage{float}

\newcommand{\incfig}[1]{%
    \def\svgwidth{\columnwidth}
    \import{./figures/}{#1.pdf_tex}
}

\usepackage{graphicx}
\graphicspath{ {./images/} }




\begin{document}


\begin{titlepage}
\includegraphics{logo}
\enlargethispage{25mm}
\addtolength{\topmargin}{-8mm}
\thispagestyle{empty}
\vspace*{\fill}
\begin{center}
      {\Huge \bf {Semestrání práce}}\\[0.2cm]
      { z předmětu ZOS}\\[0.1cm]
      {\large \bf Virtuální souborový systém}\\[0.5cm]
      
      {\Large Jan Rádl}\\[0.4cm]
\end{center}
\vspace*{\fill}
\begin{flushright}
\vfill
\today
\end{flushright}
\end{titlepage}

\tableofcontents




\chapter{Zadání}
\section{Téma}
Tématem semestrální práce bude práce se zjednodušeným souborovým systémem založeným na
i-uzlech. Vaším cílem bude splnit několik vybraných úloh.
Základní funkčnost, kterou musí program splňovat. Formát výpisů je závazný.
Program bude mít jeden parametr a tím bude název vašeho souborového systému. Po spuštění bude
program čekat na zadání jednotlivých příkazů s minimální funkčností viz níže (všechny soubory
mohou být zadány jak absolutní, tak relativní cestou)
\begin{itemize}
\item Maximální délka názvu souboru bude 8+3=11 znaků (jméno.přípona) + {\textbackslash}0 (ukončovací znak v (C/C++), tedy 12 bytů.
 \item Každý název bude zabírat právě 12 bytů (do délky 12 bytů doplníte \textbackslash0 - při kratších názvech)
\end{itemize}
\section{Příkazy}
\begin{enumerate}
 \item cp - kopíruje soubory
 \item mv - přesouvá soubory
 \item rm - maže soubory
 \item ln - vytvoří hardlink na soubor
 \item cat - vypíše obsah souboru jako sekvenci charů
 
 \item mkdir - vytváří adresář
 \item rmdir - ruší prázný adresář
 \item ls - vypíše obsah adresáře
 

 \item cd - změní aktuální adresář
 \item pwd - vypíše cestu od root adresáře k aktuálnímu adresáři
 \item info - vypíše informace do daném i-uzlu
 
 \item incp - nahraje soubor do vfs
 \item outcp - vytvoří kopii souboru z vfs do domovkého souborového systému
 \item load - začne vykonávat příkazy ze zadaného souboru
 \item format - provede zformátování vfs na požadovanou velikost v MB
\end{enumerate}





\chapter{Analýza úlohy}
\section{I-uzlový souborový systém}
\subsection{I-uzel(i-node)}
Základní jednotka souborového systému obsahující všechny podstatné informace o datech souboru nikoliv však jeho jméno.Jmenovitě:
\begin{itemize}
 \item unikátní identifikátor i-uzlu 
 \item typ souboru (sobor/adresář)
 \item velikost soboru
 \item počet odkazů ukazující na tento soubor
 \item kolekce ukazatelů na data
\end{itemize}
\subsubsection{Přímé adresování}
I-uzel obsahuje přímo adresu data bloku.
\subsubsection{Inline adresování}
Nekteré moderní soborové systémy dovolují malé množstvý dat uložit přímo v i-uzlové struktůře místo data bloku. Tento styl je pro tuto implementaci nevhodný kvůli poměrně malé velikosti i-uzlu.
\subsubsection{Nepřímé adresování n řádu}
Obdobně jako u nepřímého adresovaní pro proměné tak i zde je uložena pouze adresa na datový block obsahující odkazy o jeden řád nižší a pokud řád dosáhne 0 tak daný odkaz opět jako u přímého adresování obsahuje data souboru.
\subsection{Adresářový zázman (Dir item)}
Je další velice důležitou součástí toho systému ukládání souboru, protože dovoluje soubory ve souborovém systému pojmemovávat. Jedná se o jednoduchou strukturu obsahují jméno souboru a unikátní identifikátor i-uzlu stímto názvem.
\subsection{SuperBlock}
Je první struktura v souboru, která obsahuje informace pro zavedení a obsluhu daného soborového systému. Určuje rozdělení paměti na 4 části.
\linebreak
Obsahuje:
\begin{itemize}
 \item celkovou velikost disku
 \item velikost data bloku
 \item počet inodu
 \item počet data bloků
 \item adresu pole bitů reprezentují použitých inodu
 \item adresu pole bitů reprezentují použitých data bloků
 \item adresa prvního i-uzlu (adresa i-uzlové části systému)
 \item adresa prvního data bloku (adresa datatové části)
\end{itemize}


\chapter{Popis implementace}
\section{Prostředí}
Pro implementaci jsem si vybral programovací jazyk C++(CPP) v jeho podobě definované v jeho standartu 17(c++17). Pro jednoduché sestavení využívám nástroj CMake. Program je členěn do 2 částí:
\begin{itemize}
 \item inodefs - knihovna, která simuluje samotný souborový systém
 \item fsterminal - slouží pro obsluhu systému pomocí příkazů
\end{itemize}

\section{Řešení}
\subsection{Struktura souboru}
Samotný soubor obsahují filesystem je rozdělen do 5 částí:
\begin{enumerate}
 \item SuperBlock
 \item I-uzlové bitové pole
 \item Data blokové bitové pole
 \item Pole I-uzlů
 \item Pole data bloků
\end{enumerate}
Takto může vypadat rozložení pro předpokladané využití souboru:\linebreak
\begin{figure}[H]
\centering
\incfig{rozlozeni_fs}
\caption{Ilustrace rozložení souborového systému}
\end{figure} 
\subsection{Výpočet rozložení}
Velikost superbloku($SB_s$) je stálá, ale ostatní bloky jsou vázené na jiné bloky nebo na využitelnou velikost souboru.
Ze zadané celkové velikosti souboru($D_s$) se odečte velikost superbloku a 2 byty propřípad, že počety bloků a inodů nebude dělitelný 8.
Dále potřebuje poměr i-uzlů k data blokům v procentech($P_{ib}$) a velikost bloku($B_s$) nacházející se v souboru \ttfamily config.hpp
\normalfont a velikost i-uzlu($I_s$). Poté se spočítá počet bloků($B_{cnt}$) pro tuto velikost podle:
\[B_{cnt} = \frac{D_s - SB_s - 2}{\frac{P_{ib}}{8} + \frac{1}{8} + P_{ib}
\cdot I_s + B_s}\]
Jednotlivé části čitatele odpovídají velikostem jednotlivých bloků po přenásobení $B_{cnt}$ v bytech.
\subsection{Bitové pole}
Občas bývá také pojmenováno bitmapa. Moje implementace využívá implementaci bitSetu v cpp pro obsluhu bitových úprav a vlastního iterátoru. Logická přestava o lineárním poli bitů se neschoduje s implementací. Reálně je toto pole je tvořeno jako posloupnost 0..n bytů, kde jsou bity v bytu jsou LSB.
\begin{figure}[H]
 \centering
 \incfig{LSB}
 \caption{Logická vs Reálná}
\end{figure}
\subsection{Alokace a přidávání}
V celém programu je hodnota 0 je považovaná jako prázná/neplatná hodnota, až na root prvek, který této hodnoty nabývá jak i-uzel tak datablock a je to jediné legální místo, které tu to hodnotu může nabývat.
\subsubsection{AddPointer}
Funkce která pro zadaný i-uzel přidá ukazatel na zadaný data block. Pokud program běží v debug konfiguraci, tak při přetečení adresovatelného postoru i-uzlem ukončí program assetem a pokud je zapnuté logování a tak errorem.Tato situace je logická chyba, která nelze nijak vyřešit, protože její vyřešení by vyžadovalo rozšíření adresovatelného prostoru. V release režimu je v i-uzlu uloženo maximum adresovatelných dat a zbytek je zapsán, ale je nedostupných.
\subsubsection{AlocateX}
Je dvojice funkcí, kde X je nahrazeno (inode/datablock), která projte příslužné bitové pole a pokud narazí na volný prvek, tak ho zabere a vrátí jeho indentifikátor.
\subsection{Dealokace a odstranění}
\subsubsection{freeX}
Obdobně jako AlocateX, ale jedná se o inverzní operaci, která navíc dané místo naplní opakujícím se znakem \textbackslash0.
\subsubsection{Remove dir item}
Je funkce pro odstranění záznamu adresáře z rodičovkého adresáře pokud daný záznam je nalezen, tak je odstraněn a nahrazen posledním záznamem v tomto bloku a pokud je to zároveň poslední záznam v tomto bloku tak tento block zůstane alokovaný, ale bude prázný. Regenerace na úrovni bloků by byla příliš nákladná a navíc by zanesla do systému problém s nespojitostí datových ukazatelů v i-uzlu.




\chapter{Uživatelská příručka}
\section{Překlad}
Pro sestavení je potřeba využít nástoje CMake a díky tomu máte tyto možnosti:
\begin{itemize}
 \item Logovani - vytvoření souboru main.log obsahující informace o běhu programu podle nastaveni \ttfamily LOGLEVEL \normalfont v \ttfamily log.hpp \normalfont
 \item Debug build - sestaveni programu s kontrolou stavu programu za běhu
 \item Release build - výchozí verze programu, která nechrání proti erroru, ale přesto mohou nastat
\end{itemize}
Příklad sestavení pro normální běh programu:

\ttfamily cmake -S <adresář projektu> -B <adresář sestavení>

\normalfont
\noindent Sestavení v debug modu:

\ttfamily cmake -DCMAKE\char`_BUILD\char`_TYPE=Debug -S <adresář projektu> -B <adresář sestavení>

\normalfont
\noindent Sestavení s logováním:

\ttfamily cmake -DLOGFLAG=ON -S <adresář projektu> -B <adresář sestavení>

\normalfont
\noindent
Poté můžeme zavolat náš systémový nástroj pro setavení projektu nad adresářem setavením.
Následně v \ttfamily adresářem setavení/app \normalfont můžeme najít sestavený program pro obsluhu souborového pomocí terminálu.
\section{Nastavení parametrů}
Věškeré technické parametry souborového systému lze před sestavením samotné aplikace. Při změnách je potřeba zachovat u veliksoti bloku dvě podnímky velikost bloku je celočíselně dělitelná velikostí typu ukazatele a zárověn datablock musí být větší než velikost 2 diritemů. Dále dříve sestavený souborouvý systém se při změnách typů stává nepřenositelný.
\end{document}
