\documentclass[12pt, a4paper]{report}
\usepackage[utf8]{inputenc}
\usepackage[IL2]{fontenc}
\usepackage[czech]{babel}
\usepackage[hidelinks]{hyperref}
\RequirePackage[a4paper,left=37mm,right=37mm,top=33mm,bottom=40mm]{geometry}

\usepackage{graphicx}
\graphicspath{ {./images/} }




\begin{document}


\begin{titlepage}
\includegraphics{logo}
\enlargethispage{25mm}
\addtolength{\topmargin}{-8mm}
\thispagestyle{empty}
\vspace*{\fill}
\begin{center}
      {\Huge \bf {Semestrání práce}}\\[0.2cm]
      { z předmětu ZOS}\\[0.1cm]
      {\large \bf Virtuální souborový systém}\\[0.5cm]
      
      {\Large Jan Rádl}\\[0.4cm]
\end{center}
\vspace*{\fill}
\begin{flushright}
\vfill
\today
\end{flushright}
\end{titlepage}

\tableofcontents




\chapter{Zadání}
\section{Téma}
Tématem semestrální práce bude práce se zjednodušeným souborovým systémem založeným na
i-uzlech. Vaším cílem bude splnit několik vybraných úloh.
Základní funkčnost, kterou musí program splňovat. Formát výpisů je závazný.
Program bude mít jeden parametr a tím bude název Vašeho souborového systému. Po spuštění bude
program čekat na zadání jednotlivých příkazů s minimální funkčností viz níže (všechny soubory
mohou být zadány jak absolutní, tak relativní cestou)
\begin{itemize}
\item Maximální délka názvu souboru bude 8+3=11 znaků (jméno.přípona) + {\textbackslash}0 (ukončovací znak v (C/C++), tedy 12 bytů.
 \item Každý název bude zabírat právě 12 bytů (do délky 12 bytů doplníte \textbackslash0 - při kratších názvech)
\end{itemize}
\section{Příkazy}
\begin{enumerate}
 \item cp - kopíruje soubory
 \item mv - přesouvá soubory
 \item rm - maže soubory
 \item ln - vytvoří hard link na soubor
 \item cat - vypíše obsah souboru jako sekvenci charů
 
 \item mkdir - vytváří adresář
 \item rmdir - ruší prázný adresář
 \item ls - vypíše obsah adresáře
 

 \item cd - změní aktuální adresář
 \item pwd - vypíše cestu od root adresáře k aktuálnímu adresáři
 \item info - vypíše informace do daném i-uzlu
 
 \item incp - nahraje soubor do vfs
 \item outcp - vytvoří kopii souboru z vfs do domovkého souborového systému
 \item load - začne vykonávat příkazy ze zadaného souboru
 \item format - provede zformátování vfs na požadovanou velikost v MB
\end{enumerate}





\chapter{Analýza úlohy}
\section{I-uzlový souborový systém}
\subsection{I-uzel}
Základní jednotka souborového systému obsahující všechny podstatné informace o datech souboru nikoliv však jeho jméno.


\chapter{Popis implementace}
\section{Řešení}

\section{Výsledky}



\chapter{Uživatelská příručka}
\section{Překlad a použití}



\section{Zajímavé příkazy}




\chapter{Závěr}

\end{document}
